\section{Introduction}
\label{sec:intro}
\nocite{(*)}

Cloud computing is emerging as a popular computing platform thanks to the
flexibility that its ability to dynamically scale resources provides to clients.
Unlike user-owned servers with fixed physical resources, in a cloud computing model,
a client can lease virtual machines based on the actual demand, which can be 
dynamically allocated to a pool of physical servers.
Cloud computing 
providers such as Amazon EC2 and Rackspace can maintain a large pool of physical
resources and offer these computing resources inexpensively by leveraging 
an economy of scale. 

In order to benefit from the flexibility and cost efficiency of cloud computing,
however, clients often need to share hardware resources with untrusted parties -
potentially their competitors or malicious users. 
The cloud computing infrastructures are therefore more 
vulnerable to security threats such as side channel and covert channel attacks,
which exploit interference in shared hardware. While cache-based 
side-channel attacks~\cite{Percival,Bernstein,amazonattack} and various protection mechanisms~\cite{partitioncache,PLcache,RPcache} have been explored in
this context, to 
the best of our knowledge, timing channels through a shared memory channel have
not been studied.  
Like cache timing channels, a memory-based timing channel attack can be 
carried out without physical access to the hardware because memory latencies
of one program depends on memory accesses from other programs sharing the
memory. 
%And as with processes 
%sharing a cache, processes sharing the same main memory and memory controller 
%can interfere affecting the scheduling and return time of other memory 
%accesses.  When two simultaneous memory requests require the same resources (in 
%either the memory controller, or in the memory itself) the memory scheduler 
%must determine when to issue each request to achieve the best performance. 

In this paper, we demonstrate that memory timing channels do exist for today's 
multi-core systems, and propose an efficient protection scheme to completely
eliminate them. In a shared memory controller, the time that one memory request
is scheduled depends on other competing requests. 
Thus, there exists a memory timing channel between multiple virtual machines
sharing memory in the cloud.  This timing channel can be 
exploited by an adversary to carry out either a side-channel attack where a
malicious VM measures its own memory timing to learn a secret in another VM, or
a covert channel attack where two colluding VMs leak information to each 
another despite restrictions on explicit communications.

In order to develop a protection scheme, we first study sources of interference
in a memory controller, and categorize them into three: queueing structure 
interference, scheduler arbitration interference, and DRAM device resource 
contention. Broadly, interference is caused by multiple VMs that access the 
memory concurrently, allowing memory requests from different VMs to affect 
the timing of others. 
The goal of the protection scheme is to eliminate memory interference among 
isolation domains, which contain one or more VMs. 

We present an approach to prevent memory interference which we 
refer to as temporal partitioning (TP). Temporal partitioning groups requests in 
queues according to the isolation domain they belong to. Then, a fixed 
time period, called a turn, is statically allocated to each domain in a time
shared fashion so that a memory controller only schedules requests from one isolation 
domain. At the end of each turn, a 
short window of time during which no memory transactions can issue is added 
to drain the in-flight transactions and prevent a timing channel caused by 
shifts in refresh operations. Because scheduling decisions are exclusively 
based on requests from one active domain during each turn and no request 
from other isolation domains can cause interference, memory timing channels
are eliminated.

Experimental results suggest that the execution time overhead for temporal 
partitioning is only $1.4\%$ on average using in-order cores, and $1.2\%$ using 
out-of-order core when two isolation domains share a memory controller running
SPEC 2006 benchmarks. 
% ED: let's not oversell. Just focus on objective facts.
%Despite its significance for 
%security and its exceptionally small performance overhead, 
Temporal partitioning also requires simple changes to the memory controller with
a small mount of additional hardware resources: a revised queueing structure, a counter, 
and a small amount of combinational logic to restrict scheduling decisions.
%This not only minimizes area 
%overhead, but simplifies verification and testing complexity.

The rest of the paper is organized as follows.
Section 2 discusses the memory timing channel problem and its importance to 
cloud computing security. Section 3 analyzes a baseline memory controller for 
timing channel violations and presents the temporal partitioning scheme.  
Section 4 evaluates the security properties and execution time overheads of 
temporal partitioning experimentally and explores the trade-off in turn length 
selection using an architectural simulator. Section 5 discusses related work. 
The paper concludes in Section 6.
