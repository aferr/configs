%*********************************************************************************
% 
% Paper Title Here 
%
%

% Use IEEE style
\documentclass[11]{jhwhw}



%***********************************************************************************
%
% Packages
%
\usepackage[normalem]{ulem}
\usepackage{graphicx}
\usepackage{caption}
\usepackage{amsmath}
%\usepackage{psfrag}
%\usepackage{fullpage}

%***********************************************************************************
%
% MACROS start
%
\newcommand{\icode}{\texttt}

%%myfigure label content caption
%\newcommand{\myfigurewide}[3]
%{\begin{figure*}[hbtp]\begin{center}#2\caption{#3}\label{#1}\end{center}\end{figure*}}
%
%%mygraphfigure label size content caption
%\newcommand{\mygraphfigurewide}[4]{\myfigurewide{#1}{\includegraphics[#2]{#3}}{#4}}
%
%%myfigure label content caption
%\newcommand{\myfigure}[3]
%{\begin{figure}[hbt]\begin{center}#2\caption{#3}\label{#1}\end{center}\end{figure}}
%
%%mygraphfigure label size content caption
%\newcommand{\mygraphfigure}[4]{\myfigure{#1}{\includegraphics[#2]{#3}}{#4}}
%
%%mygraphsubfigure label size content caption
%\newcommand{\mygraphsubfigure}[4]
%{\subfigure[#4]{\hspace{0.5cm}\includegraphics[#2]{#3}\label{#1}\hspace{0.5cm}}}
%
%\newtheorem{theorem}{Theorem} % SMALL
%\newtheorem{definition}{Definition} % SMALL
%\newtheorem{problem}{Problem} % SMALL
%\newtheorem{lemma}{Lemma} % SMALL


%***********************************************************************************
%
% Anonymization Macros
%
\author{Andrew Ferraiuolo}
\title{CS5120 Homework 3}

% Usage : \ifanonymized{anonymized}{notanonymized}

% Uncomment one of the following

%\newcommand{\ifanonymized}[2]{#1} % Anonymized

%***********************************************************************************
%
% MACROS end
%

\begin{document}  

\problem{}
For each of the following Cubex statements, give typing contexts $\Delta$ and 
$\Gamma$ (assume $\Psi=\Psi_{0}$ and the other contexts are empty) in which it 
type checks with some return type and outgoing context, or else explain why it 
can't be done.
\begin{enumerate}
    \item \icode {if (x + 2 == 4) return x;}
    \item \icode {x := 0; while (f(x,y)) x := x + f(y,x);}
    \item \icode {return [b, get([3, f], y)];}
\end{enumerate}
\solution
\part
%\[
%    \dfrac{
%        \dfrac{\Gamma=\icode{x:Integer<>}}{
%            \Psi|\Theta|\Delta|\Gamma,\hat{\Gamma}|\vdash
%            \icode{x:Integer<>}
%        }
%        \quad
%        \dfrac{}{
%            \Psi|\Theta\vdash
%            \icode{Integer<>.plus:<>(2:Integer<>):Integer<>}
%        }
%    }{
%        \Psi|\Theta|\Delta|\Gamma,\hat{\Gamma}|\vdash
%        \icode{x.plus(2)} : \icode{Integer<>}
%    }
%\]
%\vspace{0.5in}
%\[
%    \dfrac{
%        \Psi|\Theta|\Delta|\Gamma,\hat{\Gamma}|\vdash
%        \icode{x.plus(2)} : \icode{Integer<>}
%        \quad
%        \dfrac{
%            \Psi=\Psi_{0}
%            \quad
%            \dfrac{}{
%                \Psi|\Theta|\Delta|\Gamma,\hat{\Gamma}|\vdash
%            4:\icode{Integer<>}}
%        }{
%            \Psi|\Theta|\vdash
%            \icode{Integer<>.equals} : <>(4:\icode{Integer<>}) : 
%            \icode{Boolean<>}
%        }
%    }{
%        \Psi|\Theta|\Delta|\Gamma,\hat{\Gamma}|\vdash
%        \icode{x.plus(2).equals(4)} : \icode {Boolean<>}
%    }
%\]
%\vspace{0.5in}
%\[
%    \dfrac{
%        \Psi|\Theta|\Delta|\Gamma,\hat{\Gamma}|\vdash
%        \icode{x.plus(2).equals(4)} : \icode {Boolean<>}
%        \quad
%        \dfrac{
%            \dfrac{
%                \Gamma = \icode{x} : \icode{Integer<>}
%            }{
%                \Psi|\Theta|\Delta|\Gamma|\hat{\Gamma}|\vdash
%                \icode{x} : \icode{Integer<>}
%            }
%        }{
%            \Psi|\Theta|\Delta|\Gamma|\hat{\Gamma}|\vdash^{true}_{Integer<>}
%            \icode{return x;} : \hat{\Gamma}
%        }
%        %        \quad
%        %        \dfrac{}{
%        %            
%        %            \icode{\{\}} : \hat{\Gamma}
%        %        }
%    }{
%        \Psi|\Theta|\Delta|\Gamma|\hat{\Gamma}|\vdash^{true}_{Integer<>}
%        \icode{if(x.plus(2).equals(4)) return x; else \{\}} : \hat{\Gamma}
%    }
%\]

In core cubex the statement is
\[
    \icode{if(x.plus<>(2).equals<>(4)) return x; else \{\}}
\]
The if statement requires that the expression in parenthesis has the 
\icode{Boolean<>} type, and that both \(return x;\) and \{\} have the same 
mutable variables available afterwards. The return statement does not make any 
new variables available so the second requirement holds. The function 
\icode{equals}, provided in the Integer class returns a boolean as needed.  
Clearly, 4 is an integer so the type context in the argument of \icode{equals} 
is correct. The expression \icode{x.plus(2)} has the type Integer, since the 
function plus provided in the class Integer returns an Integer. This requires 
that in the type context of the statment \icode{x} is an integer. Therefore, 
\(\Delta=\emptyset\) and \(\Gamma=\icode{x:Integer<>}\)

\part

From the assignment \icode{x:=0;} \icode{x} must have the type 
\icode{Integer<>}. The while statement requires that \icode{f(x,y)} has the 
type \icode{Boolean<>}. Since \icode{x} is an integer and the result of 
\icode{plus(f(y,x)} is assigned to it \icode{f(y,x)} must have the type 
\icode{Integer<>}. Therefore \icode{f} is overloaded, and to distinguish 
between these calls to different functions with the same name \icode{y} must 
have a different type from \icode{x}. Therfore,

\begin{flalign*}
    &\Delta=\icode{f<>(v1:Integer<>,v2:$\tau$):Boolean<>}, 
    \icode{f<>(v1:$\tau$,v2:Integer<>):Integer<>} &\\
                                                  & \Gamma=x:\icode{Integer<>},y:\tau &\\
                                                  & \tau \neq \icode{Integer<>} &\\
\end{flalign*}

\part

The expression \icode{[b, get([3, f], y)]} forms an \icode{Iterable<$\tau$>} of 
some type $\tau$. Each element of the iterable must have the same type, 
including the return of the function \icode{get}. Since the innermost 
Iterable contains 3, \icode{f} must also be an integer. Therfore,

\begin{flalign*}
    &\Delta=\icode{get<>($v_1$:Iterable<Integer<>>,$v_2:\tau_1$):$\tau$} &\\
    & \Gamma=\icode{f:Integer<>,b:$\tau$,y:$\tau_1$} &\\
\end{flalign*}

\problem{}
In context $\Psi = \emptyset$ and $\Theta = A, B$, show the full typing derivation proving that 
\icode{Iterable<A$\cap$B>} is a subtype of \icode{Iterable<A>$\cap$Iterable<B>}.
\solution
\[
    \dfrac{
        \dfrac{
            \dfrac{
                \dfrac{}{\Psi|\Theta\vdash\icode{A <: A}}
            }{
            \Psi|\Theta\vdash
                \icode{A$\cap$B <: A}
            }
        }{
            \Psi|\Theta\vdash
            \icode{Iterable<A$\cap$B> <: Iterable<A>} 
        }
        \quad
        \dfrac{
            \dfrac{
                \dfrac{}{\Psi|\Theta\vdash\icode{B <: B}}
            }{
            \Psi|\Theta\vdash
                \icode{A$\cap$B <: B}
            }
        }{
            \Psi|\Theta\vdash
            \icode{Iterable<A$\cap$B> <: Iterable<B>} 
        }
    }{
        \Psi|\Theta\vdash
        \icode{Iterable<A$\cap$B> <: Iterable<A>$\cap$Iterable<B>}
    }
\]
\end{document}

